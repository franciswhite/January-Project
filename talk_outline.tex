\documentclass[10pt,a4paper]{article}
\usepackage[utf8]{inputenc}

\title{%
  Cool Logic Talk: Game Theory, Pragmatics and Insults \\
  \large Haukur, Silvan}

\usepackage{mathptmx} % "times new roman"
\usepackage{amssymb}
\usepackage{amsmath, amsthm}
\usepackage{amsfonts}
\usepackage{enumitem}
\usepackage{verbatim}
\usepackage{hyperref}
\usepackage{comment}
\usepackage[margin=1in]{geometry}

\usepackage[normalem]{ulem}
\date{}
\begin{document}
\maketitle
\section{Introduction}
\begin{enumerate}
\item Talk about game theoretic approaches to pragmatics: why?\\
Some features of linguistic interaction lend themselves to GT modelling: multiple agents (talkers), picking their moves (linguistic expressions, i.e. words, sentences etc.), payoffs given for successful coordination (transfer of information. "Hey, there is a tiger behind you. - "Oh thank you, Sir, thanks for telling me, that's relevant information." "No problem, just let me know when another approaches." etc.)\\
So, the hope is that game theory can shed some light on the way we use language.
\item Give very brief and hopelessly inadequate introduction to GT and pragmatics\\
David Lewis: Communication akin to Coordination game 
\begin{table}[h]
\centering
\caption{A Simple Non-Linguistic Coordination Problem}
\label{my-label}
\begin{tabular}{|l|l|l|}
\hline
                  & Cool Handshake, yo & Serious Handshake \\ \hline
Cool Handshake, yo    & 1,1            & -5,-,5            \\ \hline
Serious Handshake & -5,-5          & 1,1               \\ \hline
\end{tabular}
\end{table}
\item Explain Signaling Game\\
Conventions are NE.\\
this doesn't explain how they arise, Lewis only explains stability. there's something normative about conventions which Lewis doesn't mention etc.
\item IBR models
\end{enumerate}
\end{document}